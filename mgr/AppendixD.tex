\addtocontents{toc}{\protect\setcounter{tocdepth}{-1}}
\chapter[Dodatek D]{Instrukcja dla projektanta}
\addtocontents{toc}{\protect\setcounter{tocdepth}{0}}
\addcontentsline{toc}{chapter}{Dodatek D}
Podobnie jak w przypadku zwykłego uruchomienia systemu, uruchamia się system w stan gotowości do pracy, tj.
\begin{enumerate}
  \item przygotowuje się węzły obliczeniowe w następujący sposób:
  \begin{enumerate}
    \item na dowolnej platformie chmurowej tworzy się węzły obliczeniowe, czyli maszyny wirtualne.
    Zaleca się użycie podobnych maszyn do użytych w toku badań, tj. maszyny z 1 vCPU, 3.5 GB pamięci ram, 7 GB pamięci masowej, system operacyjny Ubuntu 17.10,
    \item dla każdej maszyny wirtualnej otwiera się wymagane do działania systemu porty TCP 4123-4124 dla połączeń przychodzących,
    \item instaluje się oprogramowanie wymagane przez system, tj. interpreter języka Python w wersji 3, instalator bibliotek pip, biblioteki TensorFlow, Keras, Numpy,
    \item pobiera się kod systemu z przygotowanego repozytorium,
    \item uruchamia się skrypt starter.sh w tle, który uruchamia odpowiednie skrypty w języku Python i zapewnia restart systemu w razie awarii.
  \end{enumerate}
  \item do pliku workers dopisuje się adresy IP węzłów obliczeniowych, po jednym na każdą linię pliku
\end{enumerate}

Następnie można przystąpić do projektowania nowych eksperymentów, czyli nowych implementacji węzła zarządzającego.
Polega to na odpowiednim wykorzystaniu gotowego systemu kolejkowego, które przeprowadza się w sposób następujący:
\begin{enumerate}
  \item w kodzie węzła zarządzającego należy utworzyć obiekt klasy \textit{WorkManager},
  \item należy przygotować listę zadań, która jest listą obiektów klasy \textit{Job} - dla każdego zadania należy ustalić osobnika klasy \textit{Individual} do uczenia, liczbę epok uczenia oraz ziarno dla generatora liczb losowych,
  \item korzystając z metody \textit{evaluate(jobs)} obiektu klasy \textit{WorkManager} należy przeprowadzić obliczenia,
  \item uzyskaną listę wyników, czyli listę obiektów klasy \textit{Individual} z uzupełnionymi wszystkimi polami należy zapisać do pliku, np. korzystając z funkcji \textit{save\_csv\_and\_history(write\_csv, results, iteration, n, m)}.
\end{enumerate}

W ten sposób można w szybki sposób dokonywać uczenia wielu sieci na równoległych maszynach.
Przykładowe pliki, na których można wzorować nowe implementacje węzłów zarządzających:

\begin{itemize}
  \item \textit{genetic\_algorithmm.py},
  \item \textit{full\_learning.py},
  \item \textit{25epoch.py}.
\end{itemize}
