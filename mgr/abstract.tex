\chapter*{Streszczenie}

Żyjemy w czasach, w których zewsząd otaczają nas duże zbiory danych i algorytmy uczenia maszynowego.

Myślą przewodnią pracy jest zastosowanie jednego algorytmu sztucznej inteligencji do lepszego doboru parametrów innego programu tego typu.
We wstępie wysunięto tezę główną, tj. że możliwe jest optymalizowanie struktury konwolucyjnych sieci neuronowych za pomocą algorytmu genetycznego.
Teza pomocnicza głosi, że istnieje związek pomiędzy wynikami sieci po jednej epoce uczenia a jej ostatecznymi osiągami.
Została ona wysunięta w związku z potrzebą oszczędności czasu i kosztów obliczeniowych, a uczenie przez jedną epokę pozwala na ich redukcję.


Następnie zgłębione zostały kluczowe dla tego tematu zagadnienia. Poczynając od zdefiniowania pojęć takich jak uczenie maszynowe, meta uczenie czy uczenie z nadzorem,
poprzez przedstawienie problemu klasyfikacji obrazów z przybliżeniem zbioru testowego CIFAR-10, przedstawienie metody k-najbliższych sąsiadów i klasyfikatora liniowego, kończąc
na metodach sztucznej inteligencji takich jak sieci neuronowe, konwolucyjne sieci neuronowe i algorytm genetyczny.

Po zredukowaniu problemu do optymalizacji struktury po jednej epoce uczenia i przedstawieniu pojedynczej sieci neuronowej jako jednego osobnika w algorytmie genetycznym oraz zdefiniowaniu operacji genetycznych
wykonywanych nań, zaprojektowano system do przeprowadzenia eksperymentu.

System został obszernie omówiony, poczynając od węzła zarządzającego, przez system kolejkowania zadań i zaprojektowany protokół komunikacji ze zdalnymi węzłami, po strukturę samego węzła obliczeniowego
i sposób zarządzania owymi węzłami. Omówiono sposób zapewnienia replikacji badań.

Następnie przedstawione są wyniki badań i testów, w tym badanie wariancji pomiaru skuteczności klasyfikacji sieci w zależności od ziarna generatora losowego, przez testowanie działania algorytmu genetycznego i systemu kolejkowego,
po właściwe badania zależności pomiędzy uczeniem krótkim i długim oraz poszukiwania optymalnej struktury. Z badań wyciągnięto wnioski popierające tezy sformułowane we wstępie.

Następnie wyniki są krótko omawiane w podsumowaniu, gdzie również proponowane są dalsze kierunki rozwoju systemu.
\\\\
\noindent
\textbf{Słowa kluczowe}: automatyzacja, sieci neuronowe, algorytmy genetyczne, sztuczna inteligencja
\\

\noindent
\textbf{Dziedzina nauki i techniki, zgodnie z wymogami OECD}: nauki inżynieryjne i techniczne, systemy automatyzacji i kontroli
\chapter*{Abstract}
We live in a time where we are surrounded by Big Data and Machine Learning Algorithms.

The main thought driving this thesis is using an artificial intelligence algorithm for choosing better parameters for another program of the same class.
In the introduction the main thesis is introduced, which states that it is possible to perform optimization of a convolutional network structure using the genetic algorithm.
A supporting thesis states that there is a connection between classification accuracy of a neural network after 1 and after more epochs of training.
The second thesis was a natural consequence of the need to preserve time and computing resources, and 1-epoch training allows that.

Key concepts crucial for proving the thesis have been covered.
Machine learning, meta learning, and supervised learning are defined as first.
Then we move on to elaborate on the image classification problem, the CIFAR-10 dataset, k-nearest neighbor method and linear classifiers.
Finally artificial intelligence concepts such as neural networks, convolutional neural networks (CNNs) and the genetic algorithm are covered.

After reducing the problem to optimizing the structure of a convolutional neural network after 1 epoch of training, defining the specimen in the genetic algorithm as a single CNN structure and specifying the genetic operations of the algorithm, a system for conducting the experiment is proposed.

Further on, the system is broadly discussed, starting with the management node, through the queuing system and a proposed communication protocol for communication with the remote nodes, arriving at the structure of a worker node and a management system for the worker nodes.
A way to ensure reproducibility of the experiment is being introduced.

Finally the experiment and tests results are discussed, including measuring the variance of the classification accuracy of a single neural network trained with different random seeds and testing if the queuing system works properly.
After that the actual experiment results are being discussed, including looking for a connection between long- and short-term training and optimizing the network structure.
The results provide insights supporting both the theses.

Finally the results are shortly discussed in the summary, where possible further system development direction and improvements are proposed.
 \\\\
\noindent
\textbf{Keywords}: automation, neural networks, genetic algorithm, artificial intelligence\\\\
\noindent
\textbf{Field    of    science    and    technology    in    accordance    with    OECD    requirements}: engineering and technology, automation  and  control  systems
