\chapter{Podsumowanie}\label{chap:summary}

Badanie zależności pomiędzy początkową wydajnością uczonych sieci a ich wynikami po pełnym uczeniu okazało się kosztowne i czasochłonne.

Na samym początku wymagało pogłębienia wiedzy o:

\begin{itemize}
  \item uczeniu maszynowym,
  \item meta uczeniu,
  \item uczeniu z nadzorem,
  \item klasyfikacji obrazków,
  \item zbiorze CIFAR-10,
  \item metodzie k-najbliższych sąsiadów,
  \item klasyfikatorach liniowych,
  \item sieciach neuronowych - ogólnie,
  \item konwolucyjnych sieciach neuronowych,
  \item algorytmie genetycznym.
\end{itemize}

Z takim przygotowaniem teoretycznym można było podejść do zaprojektowania systemu do przeprowadzenia badań.
To z kolei wiązało się z kolejnymi krokami:

\begin{enumerate}
  \item redukcji problemu, to jest dokładniejszego zdefiniowania problemu z uwzględnieniem ograniczonego czasu i budżetu na obliczenia,
  \item zdefiniowania osobnika dla algorytmu genetycznego jako sieci neuronowej o określonej strukturze z pewnymi zmiennymi parametrami,
  \item zdefiniowaniu konkretnych operacji genetycznych jakie przeprowadzane będą w tej konkretnej implementacji algorytmu genetycznego, a konkretniej zdefiniowanie:
  \begin{itemize}
    \item krzyżowania,
    \item mutacji,
    \item upewnienia się, że osobniki znajdują się wewnątrz zdefiniowanej przestrzeni poszukiwań,
    \item substytucji między pokoleniami.
  \end{itemize}
  \item zaprojektowaniu ogólnej architektury oprogramowania do przeprowadzenia eksperymentu,
  \item zdefiniowaniu sposobu pracy węzła zarządzającego przeprowadzaniem eksperymentu,
  \item zaprojektowaniu systemu kolejkowania zadań w celu przyspieszenia obliczeń,
  \item stworzeniu protokołu do komunikacji z węzłami zdalnymi,
  \item implementacji węzła obliczeniowego wykonującego zlecone mu zadania i zwracającego wyniki,
  \item automatyzacji tworzenia nowych węzłów obliczeniowych,
  \item analizie możliwości replikacji wyników badań ze względu na korzystanie z generatora liczb pseudolosowych.
\end{enumerate}

Powstawanie owego systemu wymagało czasu, dogłębnego przemyślenia oraz pochłonęło zasoby na zarówno udane jak i nieudane próby przeprowadzenia eksperymentu.
Ostatecznie testy i wyniki, które zostały przeprowadzone można podsumować następująco:

\begin{itemize}
  \item przeprowadzono badanie wariancji wyników dla sieci o jednej strukturze przy rożnych wartościach ziarna generatora losowego w celu wyznaczenia progu poprawy pomiędzy iteracjami,
  \item przeprowadzono test działania algorytmu genetycznego dla testowej funkcji Levy'ego, na podstawie którego wyciągnięto wniosek, że algorytm genetycznie działa poprawnie i można go zastosować do właściwego zadania zmieniając wyznaczaną funkcję celu,
  \item przeprowadzono test działania systemu kolejkowego dla jednej maszyny lokalnej, z którego wynikła konieczność zmiany długości bufora dla przesyłu danych pomiędzy węzłami
  \item przeprowadzono poszukiwanie optymalnej struktury sieci neuronowej za pomocą algorytmu genetycznego w dwóch próbach, gdzie na podstawie pierwszej naprawiono błędy w systemie, a obserwując wykresy wyciągnięto wniosek, że uzyskiwana jakość sieci polepsza się (dowód tezy głównej)
  \item porównano wyniki dla uczenia 1- i 10- epokowego i na tej podstawie stwierdzono, że istnieje pomiędzy nimi związek (dowód tezy pomocniczej)
  \item zaobserwowano proces uczenia dla wybranych z pierwszej i ostatniej iteracji algorytmu genetycznego i porównano je, po raz kolejny obserwując polepszenie jakości sieci, a zatem optymalizację struktury (dowód tezy głównej)
  \item uzyskaną w wyniku działania algorytmu genetycznego sieć uczono przez 25 epok jako pełne uczenie i wyciągnięto wniosek, że proponowane stałe elementy struktury sieci mogłyby być lepiej dobrane
\end{itemize}

W toku przeprowadzonych badań udało się potwierdzić obydwie stawiane we wstępie tezy, tj.

\begin{enumerate}
  \item możliwa jest optymalizacja struktury konwolucyjnych sieci neuronowych za pomocą algorytmu genetycznego (teza główna),
  \item istnieje związek pomiędzy skutecznością sieci po krótkim i długim uczeniu (teza pomocnicza).
\end{enumerate}

Tym samym cel pracy został zrealizowany.

System można w dalszym ciągu dopracować.
Poniżej przedstawiono kilka propozycji ulepszeń:

\begin{itemize}
  \item aby zapewnić możliwość replikacji badań zrezygnowano z równoległych obliczeń na GPU i system dostosowany został do obliczeń na CPU, można by jednak uwzględnić taką możliwość w kolejnych generacjach systemu,
  \item system kolejkowy można by dodatkowo usprawnić implementując algorytm rozdzielania zadań dla równoległych maszyn, np. implementując algorytm CDS (Campbella, Dudeka i Smitha),
  \item można w dalszym ciągu ulepszyć czytelność kodu przepisując odpowiednie fragmenty,
  \item algorytmy parametry genetycznego dobrane zostały metodą prób i błędów - można by wejść na kolejny poziom meta uczenia również dobierając jego parametry za pomocą innego algorytmu uczenia maszynowego,
  \item ustalone elementy sieci neuronowej (stałą część jej struktury) można by ulepszyć tak, by ostateczna wynikowa sieć miała lepsze wyniki klasyfikacji,
  \item węzły zdalne w żaden sposób nie sprawdzają, kto wysyła im zadania.
  Może to doprowadzić do sytuacji gdzie uruchomienie jednocześnie kilku systemów kolejkowych zlecających zadanie uniemożliwia poprawne działanie systemu.
  Należałoby to poprawić w przypadku korzystania z systemu przez więcej niż jedną osobę.
\end{itemize}

Dodatkowo pojawiło się nowe pytanie: jaki wpływ ma początkowa różnica w skuteczności klasyfikacji przez sieć jednoepokową na tę samą różnicę dla 10-epok?
Widać zatem, że można dalej prowadzić badania w tym kierunku i pozostało wiele pola do badań w dziedzinie dobierania optymalnej struktury konwolucyjnuych sieci neuronowych.
