\addtocontents{toc}{\protect\setcounter{tocdepth}{-1}}
\chapter[Dodatek C]{Instrukcja obsługi systemu}
\addtocontents{toc}{\protect\setcounter{tocdepth}{0}}
\addcontentsline{toc}{chapter}{Dodatek C}

System uruchamia się w sposób następujący:
\begin{enumerate}
  \item Przygotowuje się węzły obliczeniowe w następujący sposób:
  \begin{enumerate}
    \item Na dowolnej platformie chmurowej tworzy się węzły obliczeniowe, czyli maszyny wirtualne.
    Zaleca się użycie podobnych maszyn do użytych w toku badań, tj. maszyny z 1 vCPU, 3.5 GB pamięci ram, 7 GB pamięci masowej, system operacyjny Ubuntu 17.10
    \item Dla każdej maszyny wirtualnej otwiera się wymagane do działania systemu porty TCP 4123-4124 dla połaczeń przychodząych
    \item Instaluje się oprogramowanie wymagane przez system, tj. intrepreter języka Python w wersji 3, instalator bibliotek pip, biblioteki TensorFlow, Keras, Numpy
    \item Pobiera się kod systemu z przygotowanego repozytorium
    \item Uruchamia się skrypt starter.sh w tle, który uruchamia odpowiednie skrypty w języku Python i zapewnia restart systemu w razie awarii
  \end{enumerate}
  \item Do pliku workers dopisuje się adresy ip węzłów obliczeniowych, po jednym na każdą linię pliku
  \item Lokalnie uruchamia się węzeł zarządający reperezntujący jeden eksperyment.
  W repozytorium pliki, które są do tego przeznaczone, to:
  \begin{itemize}
    \item \textit{genetic\_algorithmm.py}
    \item \textit{full\_learning.py}
    \item \textit{25epoch.py}
  \end{itemize}
  \item dokonuje się analizy zebranych w plikach csv danych przy pomocy pozostałych zgromadzonych w repozytorium skryptów
\end{enumerate}
W wyniku takiego działania otrzymuje się wyniki i wykresy identyczne jak dla przeprowadzonych w rozdizale \ref{chap:tests} badaniach, poza opisanymi błędnymi próbami, które zostały poprawione w toku pracy nad systemem.
W celu dojścia do błędnych prób zaleca się przejrzenie historii systemu kontroli wersji \textit{git} dla repozytorium \textit{gacnn} w serwisie \textit{GitHub} pod adresem \url{https://github.com/opiechow/gacnn}.
