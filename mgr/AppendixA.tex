\addtocontents{toc}{\protect\setcounter{tocdepth}{-1}}
\chapter[Dodatek A]{Skrótowy opis dyplomu}
\addtocontents{toc}{\protect\setcounter{tocdepth}{0}}
\addcontentsline{toc}{chapter}{Dodatek A}
\section{Tytuł dyplomu}
Optymalizacja struktury konwolucyjnych sieci neuronowych za
pomocą algorytmu genetycznego
\section{Cel i przeznaczenie aplikacji}
Stworzenie systemu umożliwiającego badanie możliwości użycia algorytmu genetycznego do znalezienia optymalnej struktury dla konwolucyjnej sieci neuronowej w sensie skuteczności klasyfikacji obrazów.
\section{Funkcjonalność}

\subsection{Spis realizowanych funkcji}
Wytworzony system realizuje następujące funkcje:
\begin{itemize}
  \item Generacja sieci konwolucyjnych o różnych w sensie ilości neuronów w poszczególnych warstwach strukturach
  \item Rozdzielanie zadań uczenia sieci przez system kolejkowy do poszczególnych węzłów obliczeniowych
  \item Uczenie sieci o zadanej strukturze, liczbie epok uczenia oraz przy ustalonym ziarnie generatora liczb losowych
  \item Przetwarzanie uzyskanych w wyniku przeprowadzonego eksperymentu danych do czytelnej formy wykresów
\end{itemize}
\subsection{Lista przykładowych zastosowań}
Przykładowe zastosowania wytworzonego systemu:
\begin{itemize}
  \item Badanie czy dany algorytm (w tym konkretnym przypadku algorytm genetyczny) może posłużyć do optymalizacji struktury sieci neuronowej
  \item Badanie zależności pomiędzy krótkim (np. przez 1 epokę) uczeniem sieci neuronowych a uczeniem długim (np. przez 10 epok)
\end{itemize}

\subsection{Szczegółowy opis działania aplikacji}
Algorytm genetyczny generuje potencjalnie optymalne struktury sieci neuronowych.
Wygenerowane sieci oceniane są pod względem skuteczności klasyfikacji obrazków na zbiorze testowym CIFAR-10.
Ostatecznie uzyskuje się optymalną pod względem struktury sieć neuronową oraz dużo danych z przebiegu algorytmu genetycznego.
Dane są następnie ręczne przetwarzane z wykorzystaniem pomocniczych skryptów w języku \textit{Python}.

\section{Ogólna architektura systemu}
System składa się z węzła zarządzającego przetwarzanymi eksperymentami i węzłów obliczeniowych, wykonujących powierzone im zadania.
Każde z nich z założenia jest komputerem z zainstalowanym interpreterem języka \textit{Python} w wersji 3 wraz z odpowiednim bibliotekami: \textit{Keras, TensorFlow, Numpy}.
System komunikuje się przez sieć przy pomocy interfejsu \textit{socket} poprzez protokół TCP/IP.

\section{Architektura sprzętu}

Projekt został w głównej mierze oparty o technologie wirtualizacji, zatem architektura sprzętu nie jest w dużej mierze istotna.
Podane zostaną sprzętowe parametry użytych jako węzły obliczeniowe maszyn wirtualnych: 1 vCPU, 3.5 GB pamięci RAM, 7 GB pamięci SSD.

\section{Architektura oprogramowania}
Oprogramowanie zostało podzielone na logicznie separowalne moduły.
Jeden z nich jest odpowiedzialny za działanie węzłów zarządzających i zawiera w sobie wszystkie pożyteczne dla takiego węzła funkcjonalności, np. system kolejkowy.
Drugi odpowiedzialny jest za pracę węzła obliczeniowego, który jest w istocie komputerem z jednocześnie pracującymi dwoma programami - jednym nasłuchującym zadań, drugim nasłuchującym zapytań o wyniki.
Trzeci moduł zawiera oprogramowanie pomocnicze, służące automatyzacji procesu tworzenia nowych węzłów obliczeniowych.

\section{Opis sposobu wytwarzania aplikacji}
\subsection{Założenia i sformułowanie zadania}
System powinien umożliwić zbadanie przystosowania algorytmu genetycznego do poszukiwania optymalnej struktury konwolucyjnej sieci neuronowej oraz zbadanie związku pomiędzy uczeniem krótkim i długim.
\subsection{Analiza problemu (z przeglądem dotychczasowych rozwiązań)}
Dokonano analizy problemu z uwzględnieniem następujących zagadnień: uczenie maszynowe, uczenie z nadzorem, metauczenie, klasyfikacja obrazów, klasyfikatory liniowe, konwolucyjne sieci neuronowe, algorytm genetyczny.
\subsection{Specyfikacje}
Specyfikacje w głównej mierze narzucone były przez ograniczony czas i budżet, zatem dążono do stworzenia systemu który wykona obliczenia małym kosztem i możliwie szybko.
\subsection{Przygotowanie projektu}
Projekt przygotowano odpowiednio przygotowując w odpowiedniej kolejności projekt systemu kolejkowego, węzła obliczeniowego oraz węzeł zarządzającego.
\subsection{Prototypowanie i implementacja}
Prototypowanie i implementacja odbyły się w podobnej kolejności, w jakiej przygotowano projekt - stworzono prototyp systemu kolejkowego, węzła obliczeniowego i zarządzającego, a implementacja postępowała wraz z kolejnymi ulepszeniami dokonywanymi na prototypach.
\subsection{Testowanie}
Testowano zarówno każdy moduł oddzielnie jak i integrację wszystkich modułów.
\subsection{Ocena aplikacji (w tym porównanie do innych rozwiązań)}
Aplikacja spełnia postawione przed nią zadanie. W porównaniu do istniejących rozwiązań jest rozszerzona o możliwość badania związku pomiędzy uczeniem krótkim i długim oraz korzysta ze zrównoleglenia obliczeń przy pomocy systemu kolejkowego.
\subsection{Wnioski i perspektywy dalszych prac}
Wnioski są następujące -
\begin{itemize}
  \item można poszukiwać struktury konwolucyjnych sieci neuronowych przy pomocy algorytmu genetycznego
  \item istnieje związek pomiędzy uczeniem krótkim i długim sieci neuronowych
\end{itemize}
Dalsze prace polegałyby na dopracowywaniu systemu do szybszego i stabilniejszego działania, projektowaniu nowych eksperymentów oraz uruchamianie ich na wytworzonym systemie oraz dokładniejszego zbadania zależności pomiędzy różnicą w wynikach po uczeniu krótkim a po uczeniu długim.
